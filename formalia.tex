% !TEX root = master.tex

\section*{Preamble}
\addcontentsline{toc}{section}{Preamble}

This is the exam set for the take-home exam in \course{}, \period{}.  This
document consists of \pageref{LastPage} pages; make sure you have them all.
Read the rest of this preamble carefully. Your submission will be graded as a
whole, on the 7-point grading scale, with an external examiner. 

% The exam set consists of $n$ practical tasks and $m$ theoretical questions.

\subsection*{Exam Policy}

This is an \emph{individual}, take-home, open-book exam. You are allowed to use
all the material made available during the course, without further citation.
Any other sources must be cited appropriately. If you proceed according to some
pre-existing solution, you must argue for its appropriateness.

The exam is \underline{\textbf{100\% individual}}. You are not allowed to
discuss the exam set with anyone else, until the exam is over.  It is expressly
forbidden:

\begin{itemize}

\item To discuss or share any part of this exam, including, but not limited
to, partial solutions, with anyone else.

\item To help or receive help from others.

\item To seek inspiration from other sources, including the Internet, without
proper citation.

\item To post any questions or answers related to the exam on \emph{any fora},
before the exam is over, including the course discussion forum on Absalon.

\end{itemize}

\emph{Breaches of this policy will be handled in accordance with the standard
disciplinary procedures}. Possible consequences range from your work being
considered \lstinline|void|, to \emph{expulsion from the
university}\footnote{\url{http://uddannelseskvalitet.ku.dk/docs/Ordensregler-010914.pdf}}.

\subsubsection*{Errors and Ambiguities}

In the event of errors or ambiguities in the exam text, you are expected to
state your assumptions as to the intended meaning in your report.
%
Some ambiguities may be intentional.

You may request clarifications by sending an email to our internal mailing
list, but do not expect an immediate reply: \mailinglist{}. Important
clarifications and/or corrections will be posted on the course bulletin board
on Absalon and an updated version of the exam text will made available on
\url{eksamen.ku.dk}. If there is no time to resolve a case, you should proceed
according to your chosen, documented interpretation.

We use the words ``must'', ``shall'', ``should'', etc. as described in RFC
2119\footnote{\url{https://www.ietf.org/rfc/rfc2119.txt}}.

\subsection*{What to Submit}

To pass the exam, you must submit both your source code (ZIP archive), and a
report (PDF).  Your report should both document your solution for the practical
tasks and answer the theoretical questions.

\emph{Your report forms the basis for our grading}. The report must be
printable, and easy to grade as printed. Your source code will serve only to
verify what you state in your report. You should:

\begin{itemize}

\item  Include \emph{all modified} source code in the appendices of your
report.

\item Give an overview of \emph{what} you modified and, inevitably, why.

\item Test your changes; explain how we can reproduce your test results.

\item Comment your source code; make it comprehensible.

\item Document and justify your assumptions and design decisions.

\end{itemize}

\noindent We also state the following formatting and reporting requirements:

\begin{itemize}

\item The report should be a printable PDF document with reasonable margins for
marking. A full solution, excluding appendices, is expected to be around 10-12 pages and must \emph{not} exceed 15 pages. Use normal font (not below 10pt) and page margins.

\item When including your source code, use the \texttt{listings} package in
\LaTeX, or similar. \textbf{Do not} include screenshots of your source
code\footnote{This is both tedious for you, and it cuts in the eye of your
reader.}.

\item High-resolution scanned versions (no photos, please) of handwritten
solutions are acceptable for the theoretical parts.

\item Package your source code in a ZIP archive, archiving \textbf{only} the
\texttt{src} directory. The source code should compile and run as submitted on
the \emph{virtual machine} that has been handed out\footnote{We do not have the
possibility test the code on your system, so make sure that is works on the
standard system.}. Please, \verb+make clean+ your code before zipping.

\end{itemize}

\subsection*{Language usage and indication}

You can write the exam in either Danish or English (but no mixing of the two).
However, we ask you to indicate your language choice in the following way. In
\verb+src/+ is located a file called \verb+language.txt+. If you write in
Danish write the text \verb+Danish+ and if you write in English write the text
\verb+English+. Either of the two must be the only text in \verb+language.txt+.

\subsection*{Submission}

You must submit via Digital Exam (\url{https://eksamen.ku.dk/}). If that fails,
you \emph{can} submit via Absalon. If all else fails, submit via e-mail to the
course coordinator directly (see below).

Please check your submission after you submit.

\subsection*{Emergency Line}

In case of emergencies, the course coordinator is your primary point of contact
during the exam: \responsible{}.

Please respect that our TAs might also have exams during the week.
